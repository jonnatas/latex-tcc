\chapter[Introdução]{Introdução}
O aprendizado colaborativo, assim como o PBL \textit{(Problem,
Project-Based Learning}), Aprendizagem entre Pares (\textit{Peer Instruction}), Pense-ParCompartilhe (\textit{Think-Pair-Share}) e Gamificação, dentre outros são estratégias adotados para estimular os alunos a se interesarem mais a buscarem conhecimento e aprender.

Na Aprendizagem colaborativa os alunos apredem através da toca de informações, com os diverentes pontos de vista sendo debatidos durante a resolução de questões, avaliações e demais questionamentos. \cite{lupion2004grupos}

\section{Métrodo de aprendizado}
O trezentos é um método de Aprendizagem ativa e colaborativa, desenvolvido pelo Prof.Ricardo Ramos Fragelli da Universidade de Brasilia, o método consiste em 


\section{Objetivo Geral}
Desenvolver um sistema \textit{web} capaz de aplicar facilmente o método 300 em qualquer disciplina.

\section{Objetivo específico}
O sistema deverá ter as funcionalidaes de:
\begin{itemize}
    \item Cadastrar/Editar/Remover alunos e Professores.
    \item Gerenciaar os grupos de alunos.
    \item Gerenciar as metas propostas aos alunos.
    \item Gerenciar as notas das provas dos alunos.
    \item Monitorar progresso dos alunos.
\end{itemize}