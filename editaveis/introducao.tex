\chapter[Introdução]{Introdução}
O aprendizado colaborativo, assim como o PBL \textit{(Problem,
Project-Based Learning}), Aprendizagem entre Pares (\textit{Peer Instruction}), Pense-ParCompartilhe (\textit{Think-Pair-Share}) e Gamificação, dentre outros são estratégias adotados para estimular os alunos a se interessarem mais a buscarem conhecimento e aprender.

Na Aprendizagem colaborativa os alunos aprendem através da toca de informações, com os diferentes pontos de vista sendo debatidos durante a resolução de questões, avaliações e demais questionamentos. \cite{lupion2004grupos}

\section{Métrodo de aprendizado}
O trezentos é um método de Aprendizagem ativa e colaborativa, desenvolvido pelo Prof.Ricardo Ramos Fragelli da Universidade de Brasilia, o método consiste em buscar aumentar o rendimento dos alunos através de atividades que promovem a colaboração entre estes alunos, formando assim grupos com base nas notas obtidas nas avaliações. Os estudantes após a realização de algumas metas definidas pelo professor refazem a avalização, podendo melhorar a nota obtida anteriormente. Como observado em \cite{ramos2017trezentos} mesmo sem alteração no modelo de ensino definido pelo professor houve um ganho considerável no índice de aprovação subindo de 50\% para 95\% com a aplicação do método 300.


\section{Objetivo Geral}
Desenvolver um sistema \textit{web} capaz de aplicar facilmente o método 300 em qualquer disciplina.

\section{Objetivo específico}
O sistema deverá ter as funcionalidades de:
\begin{itemize}
    \item Cadastrar/Editar/Remover alunos e Professores.
    \item Gerenciar os grupos de alunos.
    \item Gerenciar as metas propostas aos alunos.
    \item Gerenciar as notas das provas dos alunos.
    \item Monitorar progresso dos alunos.
\end{itemize}